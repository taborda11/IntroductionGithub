\documentclass[hyperref={pdfpagelayout=SinglePage}, xcolor=dvipsnames, xcolor=table]{beamer} 

\setbeamertemplate{enumerate items}[default]

\usetheme{Berlin}                             % tema
\usecolortheme{whale}                    % subtema
\renewcommand\tabcolsep{4pt}		 % diminuir espaçameto entre colunas de tabelas
\usepackage{textpos}                       % package para posicionar o logotipo do ISCTE nos slides
\usepackage{color}                           % package para alterar as cores do template beamer
\usepackage{wasysym}                     % package para adicionar novos simbolos ao ambiente 
\usepackage{csquotes}                     % package para citações  
\usepackage{wrapfig}                       % permitir imagens envolvidas em texto
\usepackage{microtype}                   % para otimizar os espaços de texto
\usepackage[labelfont=bf]{caption}   % título das tabelas e imagens em bold
\usepackage{float}
\usepackage{hyperref}
\usepackage{amsmath}
\newcommand{\dd}[1]{\mathrm{d}#1}
\usepackage{amsfonts}
\usepackage{amssymb}
\usepackage{amsbsy}
\usepackage{graphicx}
\usepackage{adjustbox}
\hypersetup{pdfpagemode=FullScreen} % para o pdf abrir automaticamente em modo full screen
\setbeamertemplate{caption}[numbered] % para numerar as tabelas
\usepackage{subcaption}
\usepackage[none]{hyphenat} 
\usepackage{bm} %%% bold vectors
\usepackage{tabularx}  
\usepackage{tikz}


\newcommand{\twopartdef}[4] %%%%%%%%%% permite fazer funções em ramos
{
	\left\{
		\begin{array}{ll}
			#1 & \mbox{if } #2 
			\\ \\
			#3 & \mbox{if } #4
		\end{array}
	\right.
}

\setbeamercolor{author in head/foot}{fg=White}     % cor do nome do autor no rodape
\setbeamercolor{institute in head/foot}{fg=White}  % cor do nome do ISCTE no rodape
\setbeamercolor{frametitle}{fg=Blue, bg=black!9} % cor da letra e do fundo da faixa com o nome de cada slide
\setbeamercolor{caption name}{fg=Blue, bg=black!9}
\setbeamercolor*{title}{fg=white, bg=Blue}          % cor do titulo e da caixa de titulo na capa
\setbeamercolor{itemize item}{fg=Blue}               % cor dos bullets do ambiente \itemize 

\makeatletter % criaçao de um ambiente sem TOC no cabeçalho do slide
\newenvironment{noheadline}{
\setbeamertemplate{headline}{}
\addtobeamertemplate{frametitle}{\vspace*{-0.9\baselineskip}}{}
}{}
\makeatother

\usepackage[beamer,customcolors]{hf-tikz}

\tikzset{hl/.style={
    set fill color=blue!80!black!40,
    set border color=blue!80!black,
  },
}

\title[Introduction to GitHub]{Introduction to GitHub}
\author[Bruno Taborda]{\textbf {Presentation for Data in Science, Bussiness and Society}} 
\titlegraphic{\includegraphics[width=10cm]{logo_1}}
\institute[ISCTE-IUL - 26-09-2019]

% position the logo
\addtobeamertemplate{frametitle}{}{%
\begin{textblock*}{100mm}(0.73\textwidth,-.85cm)
\includegraphics[scale=.05]{logo_1}
\end{textblock*}}

%%%%%%%%%%%%% Ambiente de teorema/definição com as cores do ISCTE %%%%%%

\makeatletter
\def\th@mystyle{%
    \normalfont % body font
    \setbeamercolor{block title example}{bg=Blue,fg=white}
    \setbeamercolor{block body example}{bg=black!9,fg=black}
    \def\inserttheoremblockenv{exampleblock}
  }
\makeatother
\theoremstyle{mystyle}
\newtheorem*{defi}{Definition}

\setbeamertemplate{defis}[numbered]


%%%%%%%%%%%%%%%% bibliografia %%%%%%%%%%%%%%%%%%%%%%%



\usepackage[
style=authoryear,
bibencoding=utf8, 
minnames=1, 
maxnames=2,
maxbibnames=3, % I changed from 99 to 3 (por causa do Benjamin et al)
giveninits=true,
terseinits=true
backref=false, 
natbib=true, 
dashed=false, 
terseinits=true, 
giveninits=true, 
uniquename=false, 
uniquelist=true, 
labeldateparts=true,
doi=false, 
isbn=false, 
eprint = false, % desativei. confirmar se é para manter!
url = false, % desativei. confirmar se é para manter!
hyperref=true, 
natbib=true, backend=biber]{biblatex}


\DefineBibliographyStrings{english}{
	backrefpage = {Cited on page},
	backrefpages = {Cited on pages},
}

% No dot before number of articles
\usepackage{xpatch}
\xpatchbibmacro{volume+number+eid}{\setunit*{\adddot}}{}{}{}

% Remove In: for an article.
\renewbibmacro{in:}{%
	\ifentrytype{article}{}{%
		\printtext{\bibstring{in}\intitlepunct}}}

% Get rid of notes in citations
%\AtEveryBibitem{\clearfield{note}}

% Get rid of months in citations
\AtEveryBibitem{\clearfield{month}}
\AtEveryCitekey{\clearfield{month}}
\setlength{\parindent}{1,3cm}
\raggedbottom

\renewcommand\bibfont{\scriptsize}
% If you have more than one page of references, you want to tell beamer
% to put the continuation section label from the second slide onwards

\setbeamertemplate{frametitle continuation}[from second]
% Now get rid of all the colours
\setbeamercolor*{bibliography entry title}{fg=black}
\setbeamercolor*{bibliography entry author}{fg=black}
\setbeamercolor*{bibliography entry location}{fg=black}
\setbeamercolor*{bibliography entry note}{fg=black}

% No icon in references listing
\setbeamertemplate{bibliography item}{}

% Get rid of pp. in citations
\DeclareFieldFormat{pages}{#1}

% Get rid of "" in citations
\DeclareFieldFormat[inbook]{title}{#1} 

% Order like Surname, Name
\DeclareNameAlias{author}{last-first}

\bibliography{references.bib}


\begin{document}

\begin{noheadline}

\begin{frame}
\vfill
\centering

\begin{beamercolorbox}[sep=8pt,center,colsep=-4bp,rounded=true,shadow=true]{title}
\usebeamerfont{title}\inserttitle\par%
\ifx\insertsubtitle\@empty%
\else%
\vskip0.25em%
{\usebeamerfont{subtitle}\usebeamercolor[fg]{subtitle}\insertsubtitle\par}%
\fi%     
\end{beamercolorbox}%

\vskip1em\par

\begin{beamercolorbox}[sep=8pt,center,colsep=-4bp,rounded=true,shadow=true]{author}
\usebeamerfont{author}\insertauthor
\end{beamercolorbox}

{\usebeamercolor[fg]{titlegraphic}\inserttitlegraphic\par}

\begin{beamercolorbox}[sep=8pt,center,colsep=-4bp,rounded=true,shadow=true]{institute}
\usebeamerfont{institute}\insertinstitute
\end{beamercolorbox}


\end{frame}
\end{noheadline}

%NUMERAÇÃO DOS SLIDES
\newcommand*\oldmacro{}%
\let\oldmacro\insertshorttitle%
\renewcommand*\insertshorttitle{%
  \oldmacro\hfill%
  \insertframenumber\,/\,\inserttotalframenumber}
%%%%%%%%%%%%%% ITEMIZE SHAPES AND COLLERS%%%%%%%%%%%%%%%%%%%%%%%%%

\setbeamertemplate{itemize item}{\color{Blue}\newmoon}
\setbeamertemplate{itemize subitem}{\color{Blue}$\blacktriangleright$}
\setbeamertemplate{itemize subsubitem}{\color{Blue}$\blacksquare$}

%%%%%%%%% TABLE OF CONTENTS SHAPES, COLLORS AND BULLETS %%%%%%%%%%%%%%%%

\setbeamertemplate{section in toc}[circle]
\setbeamercolor{section number projected}{bg=Blue,fg=white}


\setbeamertemplate{subsection in toc}{%
\leavevmode\leftskip=5.65ex%
\llap{\raisebox{0.2ex}{\textcolor{structure}{\color{Blue}$\blacktriangleright$}}\kern1ex}%
\inserttocsubsection\par%  
}

\begin{noheadline}

\begin{frame}[allowframebreaks]{Table of Contents}

\tableofcontents

\end{frame}

\end{noheadline}


\section{Introduction}

\subsection{What is GitHub?}
\begin{frame}{What is GitHub?}
    \begin{itemize}

        \item Open-source repository hosting service (\href{https://github.com/}{https://github.com/});
        \item Hosts your source code in multiple programming languages;
        \item Hosts your documentation;        
    \end{itemize}
    \centering
    \includegraphics[scale=.3]{github}
\end{frame}

\subsection{Why use Github?}
\begin{frame}{Why use Github?}
\begin{itemize}
    \item Easy to contribute;
    \item Easy documentation;
    \item Keep tracking changes in files;
    \item Variety of integration options (e.g. slack).
\end{itemize}
\end{frame}

\subsection{Good pratices}
\begin{frame}{Good pratices}
\begin{itemize}
    \item Store only non-compiled files (e.g. .py) and non-vendor files (e.g. .txt);
    \item Every action need to be related with a issue;
\end{itemize}
\end{frame}

\end{document}











